\documentclass[12pt]{article}
\usepackage[margin=1in]{geometry} 
\usepackage{amsmath,amsthm,amssymb,amsfonts}
\usepackage{multicol}
\usepackage{graphicx}
\usepackage{wrapfig}
\usepackage{listings}
\usepackage[portuguese]{babel}
\usepackage[utf8x]{inputenc}
\usepackage[T1]{fontenc}

\title{
\textbf{MAC0216 - Técnicas de Programação I \\ Relatório - Aventura! }
}
\author{
\textbf{Nome:} Evandro Nakayama Mota 
\hspace{2cm}
\textbf{Número USP:} 10737230 \\
\textbf{Nome:} Gabriel Brandão de Almeida
\hfill
\textbf{Número USP:} 10737182 \\
\textbf{Nome:} Leonardo Alves Pereira
\hfill
\textbf{Número USP:} 10737199 \\
}
\date {30 de Setembro de 2018}

\begin{document}
\maketitle
\thispagestyle{empty}

O programa foi implementado em módulos, responsáveis por controlar a tabela de símbolos ($hash$ $table$) e as listas ligadas.

Para inserir elementos na tabela utiliza-se a função de hash, que retorna um índice na lista a partir da soma dos valores em ASCII de cada caracter da string 'chave' módulo tam, onde tam é o tamanho da tabela. 

O tratamento de colisões na inserção é feito através da lista ligada, assim, cada posição da tabela pode possuir uma lista de elementos.

Vale ressaltar que a função retiraL, que remove um elo na lista ligada, foi implementada de forma a retirar o elo seguinte àquele que recebe como argumento.

Para realizar os testes, fizemos um programa de simulação que verifica o funcionamento das funções definidas. Para tanto, foi implementada uma função criaElemento que recebe uma string como argumento e retorna um ponteiro para um Elemento, os quais são inseridos, buscados e retirados da tabela de símbolos. Todos essas verificações são impressas na tela com mensagens de sucesso ou falha.

Para facilitar a compilação do programa fizemos uma biblioteca estática que agrupa o arquivo objeto dos módulos de listas ligadas e da tabela de simbolos. Assim, basta utilizar o comando:


\begin{verbatim}
gcc -o testes testes.c libestat.a
\end{verbatim}
\end{document}